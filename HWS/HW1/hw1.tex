\documentclass[11pt]{article}
\usepackage{enumerate}
\usepackage{url}
\usepackage{listings}
\usepackage{enumitem}
\usepackage{upquote,textcomp}

\setlength{\parindent}{0cm}

\setlength{\parskip}{0.3cm plus4mm minus3mm}

\oddsidemargin = 0.2in
\textwidth = 6.5 in
\textheight = 9.8 in
\headsep = -1in

\lstset{frame=tb,
  language=,
  aboveskip=3mm,
  belowskip=3mm,
  showstringspaces=false,
  columns=flexible,
  keepspaces=true,
  basicstyle={\small\ttfamily},
  numbers=none,
  numberstyle=\tiny\color{black},
  keywordstyle=\color{black},
  commentstyle=\color{black},
  stringstyle=\color{black},
  breaklines=true,
  breakatwhitespace=true,
  tabsize=3
}

\title{Database Systems, CSCI 4380-01 \\
Homework \# 1 \\
Due Monday September 14, 2020 at 11:59:59 PM}
\date{}
\begin{document}
\maketitle

\vspace*{-0.7in}

\noindent{\bf Homework Statement.} This homework is worth 5\% of your
total grade. If you choose to skip it, Midterm \#1 will be worth 5\%
more. Remember, practice is extremely important to do well in this
class. I recommend that not only you solve this homework, but also
work on homeworks from past semesters. Link to those is already
provided in Teams, which I am repeating here:

\url{http://cs.rpi.edu/~sibel/DBS_Past_Materials/}

This homework aims to test relational algebra first and foremost, and
a bit of normalization theory.

Please do the parts in sequence. The questions get
harder and build on your knowledge of relational algebra from previous
parts. Each question is equal weight.

{\bf Database Description.} Now that we have confused you all with so
many sites for so many different purposes, I assume you would like to
construct a website to help manage your remote learning
lifestyle. Here is a starter database idea:

\noindent {\tt Sites(\underline{sitename, username}, password, bestbrowser)}\\
\noindent {\tt Classes(\underline{classcode}, classname, semester, year, credits)}\\
\noindent {\tt Classmeetings(\underline{classcode, dayofweek, starttime}, duration, sitename, url, note)}\\
\noindent {\tt Exams(\underline{classcode, examname}, examdate,
  starttime, sitename, url, note)}\\
\noindent {\tt Teaches(\underline{classcode, instructorname}, email, note)}\\
\noindent {\tt Officehours(\underline{classcode, dayofweek, starttime}, duration, sitename, username)}\\
\noindent {\tt Resources(\underline{rid}, classcode, resourcetype, sitename, username, url)}\\

where {\tt resourcetype} is one of \verb+'discussions','groups','hw','exercises','coursenotes','videos'+.

Keys for each relation are underlined.

{\tt Sites} are sites you have account on such as Submitty, Piazza,
Gradescope, Webex Teams, etc. You must store your username. You can
also store your password in an encoded format!

{\tt Classes} are classes you are currently taking or have taken in the past. 

{\tt ClassMeetings} are the specific meetings for that class,
including lectures and study group meetings for each. For each day and
start time, you store the duration of the meeting, the site that you
use for the meeting, direct url and additional notes.

{\tt Exams} stores for each class, the date, start time, site name,
direct url and any additional notes. Examname are values like {\tt
  exam1, exam2, final}.

{\tt Teaches} stores all instructors for a course and the email that
they use for that course. We also added a note so that you can note
all weird habits and requests of each instructor.

{\tt OfficeHours} for each class are stored for each day of the week
(\verb+'Monday','Tuesday',..+), start time, duration, site and
username to go for them.

{\tt Resources} is the most useful relation in the database. It stores
where you go for each different part of a class (hw, exercise,
discussions). You store which site to go to, which username for that
site to use and the direct URL for it.

Note: All date fields are formatted as \verb+mon-day-year+,
e.g. \verb+01-31-2020+. You can assume that you can check if a date
value {\tt X} comes after another value {\tt Y} by checking whether
{\tt X $>$ Y}. Direct URLs link to the specific URL for a specific
site. For example, Submitty is a site but the direct URL for this
course on Submitty is different than the direct URL for another
course.

  \newpage

{\bf Question 1.}
Write the following queries using relational algebra. You may use any
valid relational algebra expression, break into multiple steps as
needed. However, please make sure that your answers are well-formatted
and are easily readable. Also, pay attention to the attributes
required in the output!


\begin{enumerate} [label=\Alph*]
\item Return {\tt classname}, {\tt examname} and {\tt examdate} of
  all exams with {\tt examdate} after November 1st of 2020 that the
  instructor named {\tt Fogg} is giving. \\\\
 \texttt{T1(classcode, instructorname, email, note1) = Teaches} \\ 
 \texttt{T2 = Classes * Exams * T1} \\
 \texttt{T3 = select\textunderscore{\string{examdate > 09-01-2020 and instructornname = `Fogg'}\string}} \\
 \texttt{Result = project\textunderscore{\string{classname, examname, examdate}\string} T3}

\item Return {\tt classcode, dayofweek and startime} of class meetings
  that conflict with an office hour for the same class (i.e. start at
  the same time on the same day). \\\\
 \texttt{T1(classcode1, dayofweek1, starttime1, duration1, \\ sitename1, username1) = Officehours} \\
 \texttt{T2 = T1 join\textunderscore{\string{classcode1=classcode and starttime1=starttime and \\ dayofweek1=dayofweek}\string } Classmeetings} \\
 \texttt{Result = project\textunderscore{\string{classcode, dayofweek, starttime}\string} T2}

\item Return {\tt classcode, classname} of all classes that meet on
  mondays ({\tt dayofweek}). \\\\
 \texttt{T1 = select\textunderscore{\string{dayofweek=`Monday'}\string} Classmeetings} \\
 \texttt{Result = project\textunderscore{\string{classcode, classname}\string} Classmeetings * Classes}

\item Return the username and best browser for all sites to be used
  for {\tt hw} in a class taught in {\tt Fall 2000} ({\tt
    classes.semester}, {\tt classes.year}). \\\\
\texttt{T1 = Classes * Resources * Sites} \\
\texttt{T2 = Select\textunderscore{\string{semester=`Fall' and year=2000}\string} T1} \\
\texttt{Result = project\textunderscore{\string{username, bestbrowser}\string} T2}

\item Find the {\tt sitename} of all sites used in all classes for
  class meetings, exams, office hours or other resources.\\\\
\texttt{T1 = project\textunderscore{\string{sitename}\string} Classmeetings} \\
\texttt{T2 = project\textunderscore{\string{sitename}\string} Exams} \\
\texttt{T3 = project\textunderscore{\string{sitename}\string} Officehours} \\
\texttt{T4 = project\textunderscore{\string{sitename}\string} Resources} \\
\texttt{Result = T1 $\cup$ T2 $\cup$ T3 $\cup$ T4}


\item Find the name of a pair of courses that have no sites in common
  for any activity, i.e. class meetings, exams, office hours or other
  resources. \\\\
 \texttt{T1(classcode1, dayofweek1, starttime1, duration1, \\sitename1, username1) = Officehours} \\
 \texttt{T2(classcode2, examname2, examdate2, starttime2, sitename2, url2, note2) = Exams} \\
 \texttt{T3(classcode3, dayofweek3, starttime3, duration3, \\sitename3, url3, note3) = Classmeetings} \\
 \texttt{T4 = T1 join\textunderscore{\string{classcode1=classcode2}\string } T2}  \\
 \texttt{T5 = T4 join\textunderscore{\string{classcode2=classcode3}\string } T3}  \\
 \texttt{T6 = project\textunderscore{\string{classcode1, sitename1, sitename2, sitename3}\string} T4} \\
 \texttt{T7(classcode2, sitename4, sitename5, sitename6) = T6} \\ 
 \texttt{Result = select\textunderscore{\string{classcode1<>classcode2 \\and sitename1<>sitename4 and sitename1<>sitename5 and sitename1<>sitename6 and sitename2<>sitename4 and sitename2<>sitename5 and sitename2<>sitename6 \\ and sitename3<>sitename4 and sitename3<>sitename5 and sitename3<>sitename6}\string} T7}
 
\item Find the code and name of courses with a single instructor in
  the database. \\\\
\texttt{T1(classcode1, instructorname1, email1, note1) = Teaches} \\
\texttt{T2 = Teaches join\textunderscore{\string{classcode=classcode1 and instructorname<>instructorname1}\string } T1} \\
\texttt{T3 = project\textunderscore{\string{classcode}\string} Classes  - project\textunderscore{\string{classcode}\string} T2} \\
\texttt{Result = project\textunderscore{\string{classcode, classname}\string} T3 * Classes } \\

\item Return the course code of all courses in {\tt Fall 2000} ({\tt
  classes.semester}, {\tt classes.year}) with office hours that are
  not on {\tt Monday, Wednesday}. Return also the start time and duration
  for each office hour. \\\\
\texttt{T1 = Classes * Officehours} \\
\texttt{T2 = select\textunderscore{\string{semester=`Fall' and year=2000 and dayofweek<>`Monday' \\ and dayofweek<>`Wednesday' }\string} T1} \\
\texttt{Result = project\textunderscore{\string{classcode, starttime, duration}\string} T2} \\
\end{enumerate}

{\bf Question 2.} For the following relations, (a) find and list the
keys, (b) check whether they satisfy BCNF, discuss why or why not, (c)
check whether they satisfy 3NF, discuss why or why not.

To show that a relation is not in BCNF or 3NF, you only need to show a
violation. To show that they are in BCNF or 3NF, check each functional
dependency and discuss why it is ok.

\begin{enumerate}[label=\Alph*]

\item $R1(A,B,C,D,E,F)$, ${\cal F} = \{AC\rightarrow DE, BD\rightarrow F \}$
    \begin{enumerate}[label=\Alph*]
    \item $ABC^+ = \{A,B,C,D,E,F\}$ \\
    keys: $[ABC]$
    \item This does not satisfy BCNF because in relation $AC \rightarrow DE$, $AC$ is neither a superkey nor is the relation trivial
    \item This does not satisfy 3NF because in relation $AC \rightarrow DE$, $AC$ is neither a superkey nor is the relation trivial, and $DE$ is not prime attribute 
    \end{enumerate}
\item $R2(A,B,C,D,E,F)$, ${\cal F} = \{ABC\rightarrow DEF,
  AB\rightarrow A, BCD\rightarrow AEF \}$
    \begin{enumerate}[label=\Alph*]
    \item $ABC^+ = \{A,B,C,D,E,F\}$ \\
    $BCD^+ = \{B,C,D,A,E,F\}$ \\
    keys: $[ABC, BCD]$
    \item This satisfies BCNF:\\
    $ABC \rightarrow DEF$: ABC is a superkey \\
    $AB \rightarrow A$: This functional dependency is trivial, given that A and any other attribute(s), would imply A(itself) \\
    $BCD \rightarrow AEF$: BCD is a superkey
    \item This satisfies 3NF, these functional dependencies are all valid under BCNF
    \end{enumerate}

\item $R3(A,B,C,D,E,F)$, ${\cal F} = \{ABC\rightarrow DE, BC\rightarrow AF \}$
    \begin{enumerate}[label=\Alph*]
    \item $BC^+ = \{B,C,A,F,D,E\}$ \\
    keys: $[BC]$
    \item This satisfies BCNF:\\
    $ABC \rightarrow DE$: ABC is a superkey \\
    $BC \rightarrow AF$: BC is a superkey
    \item This satisfies 3NF, these functional dependencies are all valid under BCNF
    \end{enumerate}
\item $R4(A,B,C,D,E,F)$, ${\cal F} = \{ABC\rightarrow DEF, BD\rightarrow A \}$
    \begin{enumerate}[label=\Alph*]
    \item $ABC^+ = \{A,B,C,D,E,F\}$ \\
    $BCD^+ = \{B,C,D,A,E,F\}$ \\
    keys: $[ABC, BCD]$
    \item This satisfies BCNF:\\
    $ABC \rightarrow DEF$: ABC is a superkey \\
    $BD\ \rightarrow A$: BD is not a superkey nor is the functional dependency trivial
    \item This satisfies 3NF:
    $ABC \rightarrow DEF$: ABC is a superkey \\
    $BD\ \rightarrow A$: on the right hand side, A is a prime attribute.
    \end{enumerate}
\end{enumerate}


{\bf SUBMISSION INSTRUCTIONS.} Submit a PDF document or a Text
document for this homework using Submitty. No other format and no hand
written homeworks please. No late submissions will be allowed.

If the Submitty for homework submissions is not immediately available,
we will announce it on Teams when it becomes available.

{\bf Help with relational algebra formatting.} While in class I have
been using a text version of relational algebra, which I have allowed
for many years for students who do not want to figure out the Greek
symbols. However, many past solutions use the more standard version
with Greek symbols. You can use either one in your solutions, but do
not mix and match. Use one consistently.

I present you with the full syntax here in both ways (as well as the
Latex symbols for it). Note that for the standard version, I simply
use the Math mode in Latex.

\begin{tabular}{l|l|l}
  Operation & Text Version & Standard Version \\ \hline
   Set Union & R union S & $R \,\cup\, S $ \\
   Set Intersection & R intersect S & $R \,\cap\, S $ \\
   Set Difference & R - S & $R \,-\, S $ \\
   Rename &   T(A,B,C) = R & $\rho_{T(A,B,C)} (R)$ \\
   Select & select\_\{C\} (R) & $\sigma_{C} (R)$ \\
   Project & project\_\{A1,..,An\} (R) & $\pi_{A1,..,An} (R)$ \\
   Cartesian product & R x S & $R \,\times\, S$ \\
   Theta-Join & R join\_\{C\} S & $R \,\bowtie_{C}\, S$ \\
   Natural Join & R * S & $R * S$ \\
  \end{tabular}

As an additional help, I format one of the queries we did in class in
the standard format below for two equivalent solutions.

Query: Find name of Marvel heroes who have starred in a movie

Solution 1 (text format):

T1(hid1, mid1) = HeroInMovie \\
Result = project\_\{hero\} (select\_\{hid=hid1\} (MarvelHeroes x T1)) 

Solution 1 (in standard format):
\begin{eqnarray*}
T1(hid1, mid1) & =  & HeroInMovie \\
Result &  = & \pi_{hero} (\sigma_{hid=hid1}\,\, (MarvelHeroes \times T1)) 
\end{eqnarray*}

Solution 2 (text format):

T1(hid1, mid1) = HeroInMovie \\
Result = project\_\{hero\} (MarvelHeroes join\_\{hid=hid1\}  T1))

Solution 2 (in standard format):
\begin{eqnarray*}
T1(hid1, mid1) & = & HeroInMovie \\
Result = \pi_{hero} (MarvelHeroes \bowtie_{hid=hid1}  T1))
\end{eqnarray*}

\end{document}

